\documentclass[10pt]{beamer}

\usetheme[numbering=none]{m}

\usepackage{booktabs}
\usepackage{tikz}

\title{Capability Ratios Predict Nothing}
\subtitle{}
\date{July 23, 2015}
\author{Robert J. Carroll \and Brenton Kenkel}

\newcommand{\fhat}[4]{\hat{f}_{#2}^{(- #3)} (#4 \,|\, X_{#1})}
\newcommand{\es}{\emptyset}
\newcommand{\tsum}{\textstyle\sum}

\begin{document}

\maketitle

\begin{frame}
  \frametitle{Central Question}

  How should we measure the expected outcome of a potential international dispute?
\end{frame}

\begin{frame}
  \frametitle{The Argument}

  How should we measure the expected outcome of a potential international dispute?

  \begin{itemize}
    \item Good measures predict well out of sample
    \item The usual proxy---the capability ratio---has almost no predictive power
    \item We can do much better with ensemble learning
  \end{itemize}
\end{frame}

\begin{frame}
  \frametitle{Roadmap}

  \begin{enumerate}
    \item Why the capability ratio is bad
    \item Building a better proxy
    \item \ldots and releasing it into the wild
  \end{enumerate}
\end{frame}

\plain{
  why the capability ratio is bad
}

\begin{frame}
  \frametitle{What is the capability ratio?}

  It starts with the CINC score.

  For country $i$ at time $t$,
  \begin{align*}
    \uncover<4->{\text{CINC}_{it}
    &=}
      \uncover<3->{\hphantom{+} \tfrac{1}{6} \times} \text{Troops}_{it} \uncover<2->{/ \tsum_j \text{Troops}_{jt}}
    \\
    &\quad
      \uncover<3->{+ \tfrac{1}{6} \times} \text{Spending}_{it} \uncover<2->{/ \tsum_j \text{Spending}_{jt}}
    \\
    &\quad
      \uncover<3->{+ \tfrac{1}{6} \times} \text{Population}_{it} \uncover<2->{/ \tsum_j \text{Population}_{jt}}
    \\
    &\quad
      \uncover<3->{+ \tfrac{1}{6} \times} \text{Urban Pop}_{it} \uncover<2->{/ \tsum_j \text{Urban Pop}_{jt}}
    \\
    &\quad
      \uncover<3->{+ \tfrac{1}{6} \times} \text{Energy Use}_{it} \uncover<2->{/ \tsum_j \text{Energy Use}_{jt}}
    \\
    &\quad
      \uncover<3->{+ \tfrac{1}{6} \times} \text{Iron+Steel}_{it} \uncover<2->{/ \tsum_j \text{Iron+Steel}_{jt}}
  \end{align*}
\end{frame}

\begin{frame}
  \frametitle{What is the capability ratio?}

  If $A$ and $B$ got in a dispute at time $t$, how likely would $A$ be to win?

  \begin{displaymath}
    \text{Capability Ratio}_{AB,t}
    =
    \frac{\text{CINC}_{At}}{\text{CINC}_{At} + \text{CINC}_{Bt}}
  \end{displaymath}
\end{frame}

\begin{frame}
  \frametitle{The Capability Ratio is Ubiquitous}

  \scriptsize
  \input{list-of-cites}
\end{frame}

\begin{frame}
  \frametitle{CINC Scores and Dispute Outcomes}

  \centering
  \only<1,3,5>{\input{fig-cinc-pts}}%
  \only<2>{\input{fig-cinc-pts-bd}}%
  \only<4>{\input{fig-cinc-pts-indiff}}%
  \only<6>{\input{fig-cinc-heat-pts}}%
  \only<7>{\input{fig-cinc-heat}}%
  \only<8>{\input{fig-cinc-heat-contour}}%
  \only<9>{\input{fig-cinc-heat-contour-indiff}}%
\end{frame}

\plain{
  building a better proxy
}

\begin{frame}
  \frametitle{And Now for Something Completely Different}

  \textbf{Capability Ratio:} 100\% \emph{a priori}

  \textbf{Our Approach:} 100\% data-driven
\end{frame}

\begin{frame}
  \frametitle{Two-Pronged Attack}

  How much can we improve on the capability ratio without collecting any new data?

  \begin{itemize}
    \item Data
    \begin{itemize}
      \item Break into components
      \item Allow variation over time
    \end{itemize}

    \item Modeling
    \begin{itemize}
      \item Use flexible algorithms --- a bunch of them
      \item Combine into a \emph{super learner} (van der Laan et al 2007)
    \end{itemize}
  \end{itemize}
\end{frame}

\begin{frame}[fragile]
  \frametitle{Predictors}

  Learning from National Material Capabilities and Militarized Interstate Disputes data.

\begin{small}
\begin{verbatim}
     A      B  Year    Outcome  A Troops  B Troops
--------------------------------------------------
   USA     UK  1854  Stalemate        21       234
    UK    USA  1856  Stalemate       382        26
    UK    USA  1858     B Wins       292        29
Russia  China  1862     A Wins       682      1000
Russia  China  1870     A Wins       738      1000
Russia  China  1871     A Wins       765      1000
     ...
Russia  Japan  2006  Stalemate      1027       260
Russia  Japan  2006  Stalemate      1027       260
\end{verbatim}
\end{small}

\end{frame}

\begin{frame}[fragile]
  \frametitle{Predictors}

  Learning from National Material Capabilities and Militarized Interstate Disputes data.

\begin{small}
\begin{verbatim}
     A      B  Year    Outcome  A %Troops  B %Troops
----------------------------------------------------
   USA     UK  1854  Stalemate      0.007      0.073
    UK    USA  1856  Stalemate      0.083      0.006
    UK    USA  1858     B Wins      0.092      0.009
Russia  China  1862     A Wins      0.145      0.214
Russia  China  1870     A Wins      0.185      0.252
Russia  China  1871     A Wins      0.143      0.187
     ...
Russia  Japan  2006  Stalemate      0.051      0.013
Russia  Japan  2006  Stalemate      0.051      0.013
\end{verbatim}
\end{small}

\end{frame}

\begin{frame}
  \frametitle{Candidate Models}

  \begin{columns}[t]
    \begin{column}{0.45\textwidth}
      \begin{block}{Estimators}
        \begin{itemize}
          \item Ordered Logit
          \item $k$-Nearest Neighbors
          \item Random Forest
          \item Neural Network
          \item Gaussian Process
          \item Support Vector Machine
        \end{itemize}
      \end{block}
    \end{column}

    \begin{column}{0.45\textwidth}
      \begin{block}{Covariates}
        \begin{itemize}
          \item Raw Components
          \item Raw Components + Year
          \item Annual Proportions
          \item Annual Proportions + Year
        \end{itemize}
      \end{block}
    \end{column}
  \end{columns}

  \vspace{-2em}
  \begin{align*}
    \qquad\qquad\qquad
    &= \text{24 Candidate Models} \\
    &\quad + \text{Null Model} \\
    &\quad + \text{Ordered Logit on Capability Ratio}
  \end{align*}
\end{frame}

\begin{frame}
  \setbeamercolor{alerted text}{fg=gray!30}
  \frametitle{The Super Learner}

  \begin{center}
  \renewcommand{\arraystretch}{1.4}

  \only<-13>{
    \begin{tabular}{cccccc}
      &&& \hphantom{$\fhat{5}{5}{5}{B}$} & \hphantom{$\fhat{5}{5}{5}{B}$} & \hphantom{$\fhat{5}{5}{5}{B}$} \\[-1em]
      $Y$ & $X$ & Fold & $\hat{f}_1$ & $\hat{f}_2$ & $\hat{f}_3$ \\
      \hline
      \alert<3-7>{$\es$} & \alert<3-7>{$X_1$} & \uncover<2->{\alert<3-7>{1}} & \uncover<3->{\only<4-6>{$\fhat{1}{1}{1}{\es}$}\only<7->{0.10} & \only<6>{$\fhat{1}{2}{1}{\es}$}\only<7->{0.40} & \only<6>{$\fhat{1}{3}{1}{\es}$}\only<7->{0.85}} \\
      \alert<3-7>{$B$} & \alert<3-7>{$X_2$} & \uncover<2->{\alert<3-7>{1}} & \uncover<3->{\only<5-6>{$\fhat{2}{1}{1}{B}$}\only<7->{0.05} & \only<6>{$\fhat{2}{2}{1}{B}$}\only<7->{0.30} & \only<6>{$\fhat{2}{3}{1}{B}$}\only<7->{0.60}} \\
      \alert<9-10>{$A$} & \alert<9-10>{$X_3$} & \uncover<2->{\alert<9-10>{2}} & \uncover<10->{0.40 & 0.70 & 0.70} \\
      \alert<9-10>{$B$} & \alert<9-10>{$X_4$} & \uncover<2->{\alert<9-10>{2}} & \uncover<10->{0.95 & 0.75 & 0.45} \\
      \alert<11-12>{$A$} & \alert<11-12>{$X_5$} & \uncover<2->{\alert<11-12>{3}} & \uncover<12->{0.30 & 0.65 & 0.05} \\
      \alert<11-12>{$\es$} & \alert<11-12>{$X_6$} & \uncover<2->{\alert<11-12>{3}} & \uncover<12->{0.10 & 0.80 & 0.80} \\
      \hline
    \end{tabular}
  }\only<14->{
    \begin{tabular}{cccccc}
      &&& \hphantom{$\fhat{5}{5}{5}{B}$} & \hphantom{$\fhat{5}{5}{5}{B}$} & \hphantom{$\fhat{5}{5}{5}{B}$} \\[-1em]
      $Y$ & $X$ & Fold & \multicolumn{3}{c}{Super Learner} \\
      \hline
      $\es$ & $X_1$ & 1 & \multicolumn{3}{c}{$p_1 = w_1 (0.10) + w_2 (0.40) + w_3 (0.85)$} \\
      $B$   & $X_2$ & 1 & \multicolumn{3}{c}{$p_2 = w_1 (0.05) + w_2 (0.30) + w_3 (0.60)$} \\
      $A$   & $X_3$ & 2 & \multicolumn{3}{c}{$p_3 = w_1 (0.40) + w_2 (0.70) + w_3 (0.70)$} \\
      $B$   & $X_4$ & 2 & \multicolumn{3}{c}{$p_4 = w_1 (0.95) + w_2 (0.75) + w_3 (0.45)$} \\
      $A$   & $X_5$ & 3 & \multicolumn{3}{c}{$p_5 = w_1 (0.30) + w_2 (0.65) + w_3 (0.05)$} \\
      $\es$ & $X_6$ & 3 & \multicolumn{3}{c}{$p_6 = w_1 (0.10) + w_2 (0.80) + w_3 (0.80)$} \\
      \hline
    \end{tabular}
  }

  \uncover<15->{Select $w$ to maximize $\displaystyle\sum_{i=1}^N \log p_i$.}
  \end{center}
\end{frame}

\begin{frame}
  \frametitle{Sample Split}

  80--20 split between training and test samples.

  \begin{center}
    \input{tab-mid}
  \end{center}
\end{frame}

\begin{frame}
  \frametitle{Ensemble Weights}

  \centering
  \input{fig-weight-hist}
\end{frame}

\begin{frame}
  \frametitle{Ensemble Weights}

  \centering
  \input{fig-weight-by-method}

  \vspace{-3em}
  \input{fig-weight-by-data}
\end{frame}

\begin{frame}
  \frametitle{Ensemble Weights}

  \centering
  \input{fig-weight-by-cvl}
\end{frame}

\begin{frame}
  \frametitle{Test Set Results}

  \centering
  \input{tab-test}
\end{frame}

\begin{frame}
  \frametitle{Test Set Results}

  \centering
  \only<1>{\input{fig-tern-all}}%
  \only<2>{\input{fig-tern-ab}}%
\end{frame}

\begin{frame}
  \frametitle{Constructing DOE Scores}

  New measure --- Dispute Outcome Expectations (DOE) scores
  \begin{enumerate}
    \item Take every pairing of countries ever

    \item Plug their covariate values into the super learner

    \item Get back $\Pr(\text{A Wins}), \Pr(\text{B Wins}), \Pr(\text{Stalemate})$
  \end{enumerate}
\end{frame}

\begin{frame}
  \frametitle{DOE vs. CINC}

  \centering
  \input{fig-usa-russia}
\end{frame}

\plain{
  into the wild:\\applying the new measure
}

\begin{frame}
  \frametitle{Articles We Wanted to Replicate (94)}

  \scriptsize
  \input{list-of-cites}
\end{frame}

\begin{frame}
  \frametitle{Articles We Actually Replicated (18)}

  \scriptsize
  \input{list-of-replications}
\end{frame}

\begin{frame}
  \frametitle{Replication Procedure}

  \begin{enumerate}
    \item Reproduce original CINC model
    \item Replace CINC scores with DOE scores and re-run
    \item Compare model fit:
    \begin{itemize}
      \item AIC
      \item Vuong (1989) test
      \item CV loss
    \end{itemize}
  \end{enumerate}
\end{frame}

\begin{frame}
  \frametitle{Replication Results}

  \centering
  \only<1>{\input{fig-replication-aic}}%
  \only<2>{\input{fig-replication-vuong}}%
  \only<3>{\input{fig-replication-prl}}%
\end{frame}

\plain{
  conclusions
}

\begin{frame}
  \frametitle{Measuring Expected Dispute Outcomes}

  \begin{itemize}
    \item Advantages of DOE scores
    \begin{itemize}
      \item Interpretability
      \item Predictive power
      \item Usually improve regression fits
    \end{itemize}

    \item Advice to applied researchers
    \begin{itemize}
      \item Don't choose measures blindly
      \item Cross-validate
    \end{itemize}
  \end{itemize}
\end{frame}

\begin{frame}
  \frametitle{Lessons for Proxy Variable Construction}

  \begin{itemize}
    \item Optimize predictive performance
    \item Use data, but not greedily
    \item Traditional hypothesis testing can benefit from atheoretical (or algorithmic) measures
  \end{itemize}
\end{frame}

\plain{
  thank you!
}

\end{document}
