\documentclass{beamer}
\usepackage{tikz}

\usetheme[numbering=none]{m}

\newcommand{\fhat}[4]{\hat{f}_{#2}^{(- #3)} (#4 \,|\, X_{#1})}
\newcommand{\es}{\emptyset}
\newcommand{\tsum}{\textstyle\sum}

\title{Capability Ratios Predict Nothing}
\date{November 3, 2015}
\author{Robert J. Carroll \and Brenton Kenkel}

\begin{document}

\maketitle

\begin{frame}
  \frametitle{What We$^*$ Want to Measure}

  \vspace{5em}

  In a hypothetical dispute between two states, how likely is each side to win?

  \vspace{5em}

  \begin{scriptsize}
    $^*$ ``We'' being empirical IR scholars.
  \end{scriptsize}
\end{frame}

\begin{frame}
  \frametitle{The Argument}

  Use \textbf{(out-of-sample) predictive power} as the criterion for measures of expected dispute outcomes.
\end{frame}

\begin{frame}
  \frametitle{How It's Done Now}

  It starts with the CINC score.

  For country $i$ at time $t$,
  \begin{align*}
    \uncover<4->{\text{CINC}_{it}
    &=}
      \uncover<3->{\hphantom{+} \tfrac{1}{6} \times} \text{Troops}_{it} \uncover<2->{/ \tsum_j \text{Troops}_{jt}}
    \\
    &\quad
      \uncover<3->{+ \tfrac{1}{6} \times} \text{Spending}_{it} \uncover<2->{/ \tsum_j \text{Spending}_{jt}}
    \\
    &\quad
      \uncover<3->{+ \tfrac{1}{6} \times} \text{Population}_{it} \uncover<2->{/ \tsum_j \text{Population}_{jt}}
    \\
    &\quad
      \uncover<3->{+ \tfrac{1}{6} \times} \text{Urban Pop}_{it} \uncover<2->{/ \tsum_j \text{Urban Pop}_{jt}}
    \\
    &\quad
      \uncover<3->{+ \tfrac{1}{6} \times} \text{Energy Use}_{it} \uncover<2->{/ \tsum_j \text{Energy Use}_{jt}}
    \\
    &\quad
      \uncover<3->{+ \tfrac{1}{6} \times} \text{Iron+Steel}_{it} \uncover<2->{/ \tsum_j \text{Iron+Steel}_{jt}}
  \end{align*}
\end{frame}

\begin{frame}
  \frametitle{How It's Done Now}

  If $A$ and $B$ got in a dispute at time $t$, how likely would $A$ be to win?

  \begin{displaymath}
    \text{Capability Ratio}_{AB,t}
    =
    \frac{\text{CINC}_{At}}{\text{CINC}_{At} + \text{CINC}_{Bt}}
  \end{displaymath}
\end{frame}

\begin{frame}
  \frametitle{Problems with the Capability Ratio}

  \begin{itemize}
    \item Not designed for dyads
    \item Unweighted
    \item Inflexible over time
    \item Why a ratio?
    \item Nil predictive power
  \end{itemize}
\end{frame}

\begin{frame}
  \frametitle{How to Do It Better}

  \begin{itemize}
    \item Learn from dispute outcome data

    \item Break CINC into components

    \item Allow variation over time

    \item Use flexible predictive algorithms and ensemble learning

    \item Cross-validate to check overfitting
  \end{itemize}
\end{frame}

\begin{frame}
  \frametitle{Our Data}

  \begin{itemize}
    \item Sample: MIDs ($N = 1{,}740$)
    \item Response: Dispute outcome
    \begin{itemize}
      \item Side A wins
      \item Stalemate
      \item Side B wins
    \end{itemize}
    \item Predictors:
    \begin{itemize}
      \item Six CINC components
      \item Annual proportions of each component
      \item Year of dispute
    \end{itemize}
  \end{itemize}
\end{frame}

\begin{frame}
  \frametitle{Modeling Strategy}

  \begin{columns}[t]
    \begin{column}{0.45\textwidth}
      \begin{block}{Estimators}
        \begin{itemize}
          \item Ordered Logit
          \item C5.0
          \item Support Vector Machine
          \item $k$-Nearest Neighbors
          \item CART
          \item Random Forest
          \item Neural Networks
        \end{itemize}
      \end{block}
    \end{column}

    \begin{column}{0.45\textwidth}
      \begin{block}{Variable Combinations}
        \begin{itemize}
          \item Raw Components
          \item Raw Components + Year
          \item Annual Proportions
          \item Annual Proportions + Year
        \end{itemize}
      \end{block}
    \end{column}
  \end{columns}

  \vspace{-2em}
  \begin{align*}
    \qquad\qquad\qquad
    &= \text{28 Candidate Models} \\
    &\quad + \text{Null Model} \\
    &\quad + \text{Ordered Logit on Capability Ratio}
  \end{align*}
\end{frame}

\begin{frame}
  \frametitle{Model Comparison}

  \input{fig-prl}
\end{frame}

\begin{frame}
  \frametitle{Super Learner}

  \input{fig-wt}
\end{frame}

\begin{frame}
  \frametitle{Comparison: In-Sample Prediction}

  \input{fig-in-sample}
\end{frame}

\begin{frame}
  \frametitle{Comparison: Out-of-Sample Prediction}

  \textbf{Capability Ratio:} 1\% improvement over null
  
  \bigskip%
  \textbf{Super Learner:} 20\% improvement over null
\end{frame}

\begin{frame}
  \frametitle{DOE Scores}

  \textbf{Dispute Outcome Expectations:}
  \begin{enumerate}
    \item Plug CINC components for every dyad-year into model
    \item Predicted probabilities = DOE scores
    \item Drop-in replacement for capability ratio
  \end{enumerate}

  (Download from \texttt{doe-scores.com}!)
\end{frame}

\begin{frame}
  \frametitle{Replications}

  \input{fig-replication-prl}
\end{frame}

\begin{frame}
  \frametitle{Summary}

  \begin{itemize}
    \item Focus on out-of-sample prediction
    \item Capability ratios predict (almost) nothing
    \item Superiority of DOE scores:
    \begin{itemize}
      \item As a predictor of dispute outcomes
      \item As a control in conflict regressions
    \end{itemize}
  \end{itemize}
\end{frame}

\begin{frame}
  \frametitle{What's Next}

  \begin{itemize}
    \item Stricter forecasting
    \item Additional variables
    \begin{itemize}
      \item Distance and contiguity
      \item Nuclear weapons
    \end{itemize}
    \item Your suggestions!
  \end{itemize}
\end{frame}

\end{document}

