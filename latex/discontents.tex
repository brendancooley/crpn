\section{The Capability Ratio and Its Discontents}

Thanks in part to the popularity of formal models of choice under uncertainty, many unobserved quantities like those discussed above take the form of probabilities.
Our application---expectations about war outcomes as parameterized by some probability $p \in [0,1]$---is no different.
We want to create a proxy for the chance that Country~A would prevail in a dispute against Country~B, given their observable characteristics, $x_A$ and $x_B$.
Since a measure of a probability must lie within the unit interval, a natural way to proceed is to propose an indexing function $g$, where $g(x) \geq 0$, and then take the ratio of indices,
\begin{equation}
  \label{eq:ratio}
  f(x_A, x_B)
  =
  \frac{g(x_A)}{g(x_A) + g(x_B)}.
\end{equation}
The quality of such a measure depends on both the selected characteristics and on the appropriateness of the indexing function~$g$.
This latter responsibility plays a large role in the development of good measures and is our primary area of focus.

Though simple, this enhanced ratio-based approach is remarkably powerful and finds use in a diverse array of applications.
A classic success comes from the study of baseball outcomes, where the Pythagorean prediction \citep{james1983,miller2007} of a team's winning percentage is defined as
\begin{align*}
  f\left(\text{Runs Scored}, \text{Runs Allowed} ; \alpha\right) &= \frac{\text{Runs Scored}^\alpha}{\text{Runs Scored}^\alpha + \text{Runs Allowed}^\alpha},
\end{align*}
where $\alpha \geq 0$ adjusts $x$'s shape.
Here the quest for the best-fitting $g$ amounts to estimating $\alpha$; \citet{james1983} originally proposed $\alpha = 2$ \emph{ad hoc}, and later analysts found that $\alpha = 1.83$ fit the data best.
Though the analyses that produced this estimate suffer from the overfitting problems discussed above, the Pythagorean predictor still performs quite well when imposed upon out of sample data.

% TODO: Some of this notation collides with what we use for multiple imputation in the appendix.  Do we care?

When proxying for expected dispute outcomes, empirical conflict scholars normally use transformations of data on states' material capabilities.
We now relate the typical transformation to our discussion of ratio-based measures above.
We begin by introducing some helpful notation: call the set of states $\mathcal{I} = \left\{1, \ldots, I\right\}$; the set of variables $\mathcal{J} = \left\{1,\ldots,J\right\}$; and the set of years $\mathcal{T} = \left\{1,\ldots, T\right\}$.
Denote state $i$'s value for variable $j$ in time $t$ as $M_{ijt}$.
The set of all data is $M$, and all data in year $t$ is $M_t$.
Define state $i$'s share of variable $j$ in year $t$ as
\begin{align*}
  S_{ijt}(M_{t}) &= \frac{M_{ijt}}{\sum_{\mathcal{I}} M_{ijt}}.
\end{align*}

We now introduce the \emph{CINC function}.\footnote{Here CINC stands for ``Composite Index of National Capability'' as given in the Correlates of War National Material Capabilities data \citep{singer1972}.}
State $i$'s CINC score in year $t$ is its average share across all variables:
\begin{align*}
  \CINC_i(M_t) &= \frac{\sum_{\mathcal{J}} S_{ijt}}{\left|\mathcal{J}\right|}.
\end{align*}
State $i$'s CINC score therefore falls in $[0,1]$.
The CINC score is the ``most commonly used measure'' of power in empirical conflict studies \citep[212]{kadera2004}.

Following the discussion in the previous section, the most intuitive CINC-based proxy for $p$ is a na\"{\i}ve \emph{capability ratio}:
\begin{align*}
  f_{\text{CR}}(M_t) &= \frac{\CINC_A(M_t)}{\CINC_A(M_t) + \CINC_B(M_t)}.
\end{align*}
Our approach, then, makes explicit the fact that $\CINC$ is simply a candidate $g$ function imposed upon annual material capability data $M_t$.

Problems emerge immediately.
Two of these pertain to the CINC function itself.
First, as has been documented, the CINC function is sensitive to changes in state membership over time \citep{organski1980,gleditsch1999,kadera2004}.
Second, the CINC function's equal weighting of all indicators is entirely \emph{ad hoc}.
For example, the CINC function assigns the same importance to military spending as it does to personal energy consumption.
Whether this is an appropriate assignment is an empirical question that goes unanswered.
Even on the tenuous assumption that $\CINC$ is a good data reduction technique on $M_t$, it is not clear whether it serves as a good $g$ function.
Prior to entering $f_{\text{CR}}$, should the CINC scores be exponentiated given some parameter on returns to scale, or perhaps instead logged?
Finally, even given that $\CINC$ is a useful index, we do not know whether a ratio-based approach is appropriate at all.
In other words, taking the capability ratio at its word requires making a host of assumptions that may not hold well enough to make it useful in applications.\footnote{
  It is worth noting that some applications follow \citet{bremer1992} in using the explicit CINC ratio:
  \begin{align*}
    f_{\text{Bremer}}(M_t) &= \frac{\max\left\{\CINC_A(M_t),\CINC_B(M_t)\right\}}{\min\left\{\CINC_A(M_t),\CINC_B(M_t)\right\}}.
  \end{align*}
  Bremer's approach does not fall in $[0,1]$, though it could be transformed through a logit or probit CDF.
  However, it is a monotonic transformation of the aforementioned capability ratio, and it suffers from similar problems anyway.
}

Yet it is widely used. Capability ratios (or similar manipulations of CINC scores) feature prominently in many recent empirical studies in international relations.
As we might expect given the importance of the bargaining model, many of these \citep[e.g.][]{gartzke2007,salehyan2008} use capability ratios in regressions predicting the onset of a militarized interstate dispute.
Still others focus on particular features of a militarized interstate dispute, such as the nature of its termination \citep{beardsley2008} or whether its combatants complied with laws of war \citep{morrow2007}.
Still other studies focus on other phenomena not directly related to disputes, such as the onset of sanctions \citep{whang2013}, issue agreements \citep{mitchell2007}, or nuclear assistance provisions \citep{kroenig2009}.
A more exhaustive survey of the use of capability ratios is beyond the scope of this paper, but suffice it to say that it is the go-to measure of relative power in international relations.

One might politely defend the capability ratio by noting that it asks the $\CINC$ function to perform a job it was not designed for.
We would agree.
It is worth noting, however, that early proponents of the $\CINC$ function \citep[24]{singer1972} sought to understand how ``uncertainty links[s] up with capability patterns on the one hand and with war or peace on the other.''
Writing over two decades before the classic introduction to the bargaining model of war \citep{fearon1995}, these authors lacked the abstract target---$p$---that we currently have, but their enterprise was largely similar.
Though their focus on systemic, rather than dyadic, patterns reflects the dominant flavor of realism at the time, they still wanted to know how preponderance of power related to the decision to fight.
So, while it remains true that the capability ratio is not meant to directly relate to $p$ in a bargaining model, it \emph{was} meant to tell us something about how uncertainty relates to war.

Remedying these problems by fitting a model---perhaps estimating an exponent on the CINC scores in $f_{\text{CR}}$, or weights for the CINC indicators---would be a laudable first step, but it would still suffer from functional form dependencies.
The careful scholar might sidestep these by fitting an ensemble of models with different functional forms imposed, but so long as the ensemble is fit to the entire capability data set, it will suffer from overfitting and lose predictive power.
In the next section, we outline our method for estimating $p$ that suffers neither from overfitting nor from pathologies in the $\CINC$-utilizing, ratio-based approach.

%%% Local Variables:
%%% mode: latex
%%% TeX-master: "doe"
%%% End:
