\documentclass[11pt,oneside]{article}

\usepackage[T1]{fontenc}
\usepackage[utf8]{inputenc}
\usepackage{amsmath}
\usepackage{sectsty}
\usepackage{tikz}
\usepackage[labelsep=period,font=small,labelfont=bf]{caption}
\usepackage[charter]{mathdesign}
\usepackage[scaled]{helvet}
\usepackage[hyphens]{url}
\usepackage[colorlinks=true,citecolor=black]{hyperref}

% Highlighted notes
\usepackage{xcolor}
\usepackage{soul}
\newcommand{\note}[1]{\hl{[#1]}}

% Math commands
\DeclareMathOperator{\CVL}{CVL} % CV loss
\DeclareMathOperator{\CINC}{CINC} % CINC function

% Section heading styling
\allsectionsfont{\sffamily}

% BibTeX
\usepackage{natbib}
\renewcommand{\harvardurl}[1]{\textbf{URL:} \url{#1}}

\title{
  Capability Ratios Predict Nothing%
  \thanks{%
    We thank Zach Jones and Marc Ratkovic for helpful discussions and advice.
    Bryan Rooney provided excellent research assistance.
    All code and data necessary to replicate the analyses here, along with a version history of the project, are available at \url{https://github.com/brentonk/doe}.
  }%
}
\author{%
  Robert J. Carroll%
  \thanks{%
    Assistant Professor, Department of Political Science, Florida State University.  Email:  \nolinkurl{RobCarrollFSU@gmail.com}.
  }%
  \and%
  Brenton Kenkel%
  \thanks{
    Assistant Professor, Department of Political Science, Vanderbilt University.
    Email: \nolinkurl{brenton.kenkel@vanderbilt.edu}.
  }%
}

\begin{document}

\maketitle

\begin{abstract}
  Most empirical studies of international conflict use a ratio of military capability index scores as a proxy for the expected outcome of war.
  Despite the near-universal inclusion of capability ratios in statistical analyses of conflict, there has been no effort to validate whether these ratios are reliable predictors of war outcomes.
  Because states' expectations about potential war outcomes play a large role in theories of conflict, including the bargaining model of war, it is crucial for empirical research to control for a valid measure of these expectations.
  We proceed in three steps.
  First, we show that the predictive performance of the capability ratio is nil: it does no better than a null model at predicting war outcomes.
  Second, we advocate a data-driven approach to constructing a superior proxy for expected war outcomes, using the train-validate-test paradigm from the machine learning literature.
  Applying this method, we find that we can develop a substantially better measure of expected war outcomes using the same components that are used to construct the capability ratio.
  Third, we use our new measure to replicate existing studies of the role expectations play in the outbreak of international conflict.
\end{abstract}

\clearpage


\section{Introduction}


\section{Predictive Power and Proxy Variables}

We are often interested in questions that link observed data to some unobserved quantity.
This latter quantity may be unobserved because it is difficult (or impossible) to measure directly (like wealth) or because it is an abstraction (like the ideal point of a voter in a spatial model).
In either case, the applied analyst faces a choice between omitting some potentially important variable and including some proxy variable in its stead \citep{stahlecker1993}.
There is no best choice: some theoretical econometricians \citep[e.g.][]{mccallum1972} argue for the inclusion of all proxies (including crude ones), while others \citep[e.g.][]{maddala1977} support only the use of reliable proxies.
Even those in the former camp, however, admit that reliable proxies perform better than unreliable ones.

Healthy disciplines use good measures for central concepts \citep{kuhn1977}, and so social science progresses, in part, by developing better ways to construct proxy variables.\footnote{Here we focus on the importance of models in producing measures; equal weight should be assigned to advances in the estimation of these models' relevant parameters, most notably to advances in Bayesian estimation \citep{jackman2001,martin2002,clinton2004,bafumi2005}.}
Much recent progress is due to the development of measurement models.\footnote{Of course, the use of theory in the act of measurement is nothing new.
  Economics retains its longstanding commitment to structural estimation whereby theoretical models are used to uncover relevant quantities.
  For current applications to the structural estimation of dynamic discrete-choice games (for example), see \citet{su2012} and \citet{egesdal2013}.} \citet[2]{jacoby2014} observes:
\begin{quote}
  ``All of us are comfortable with the notion of statistical \emph{models} that provide representations of structural relationships between variables.  But, modern social science also regards measurement as a model that pertains to each of the individual variables.  Careful attention and rigorous approaches are just as important for the latter type of models, as they are for the former.''
\end{quote}
Moreover, when the unobserved quantity is an abstraction, appropriate measurement models allow the analyst to perform direct tests that follow from the same set of assumptions as those used in the original, theoretical model.
As \citet[355]{clinton2004} put it in the context of testing legislative behavior, ``it is inappropriate to use ideal points estimated under one set of assumptions...to test a different behavioral model....''  \citet[530]{shor2011} note that the close relationship between statistical and theoretical models of legislative behavior (and their requisite assumptions) ``has contributed to a much tighter link between theory and empirics in these subfields of political science.''

While better models (and better ways to estimate their parameters) have improved our measures of a variety of important quantities, it remains problematic that modern political measurement has ignored the importance of predictive power in producing proxies.
This is odd, as seminal contributions to the literature utilize classification---a criterion often used in machine learning, where the focus is usually on prediction---as a way to prove a new measure's superiority over extant ones. For example, in the classic paper on ideal point estimation in American legislatures, \citet[Table 3]{poole1985} report that a simple classification approach based on their NOMINATE scores correctly predicts over 80\% of legislative votes in most years in their data. Though their procedure estimates ideal points via the method of maximum likelihood rather than via a classification criterion, it remains that this analysis lies prone to textbook overfitting problems.
In their paper, Poole and Rosenthal estimate ideal points within a single Congressional session, and their classification test then uses those ideal points to assess voting within the same Congressional session.
While many of the correct classifications reflect the spatial model's explanatory virtues, others may arise due to overfitting to the data within that Congressional session.

It is for these reasons that we advocate a data-driven approach based on predictive performance.  While traditional statistical approaches to measurement minimize error or maximize likelihood within the entire data set, we instead aim to optimize out-of-sample prediction.\footnote{Our formal criterion for predictive performance, the log loss function, is explicitly described in the methods section.}
We do so through cross-validation.
Each observation in our data is assigned to a ``fold,'' and each fold is given a battery of predictions based on models fit to data not including the fold.  
That is, each observation is assigned a set of predictions from models fit to other data.
Our battery of predictions include a variety of estimators, including traditional models like ordered logit but also more flexible algorithms like random forests or averaged neural nets. 
We determine the best weighted average of these estimators by using the out-of-sample predictions assigned to each observation.
With models fit, predictions assessed, and weights derived, we use the results to produce a final estimate.

This approach explicitly addresses the problems enumerated above: it avoids the overfitting problems associated with traditional measurement techniques and the model selection problems associated with attempts to bring new data to bear for predictive purposes.
We believe that the costs of our approach---additional computation and difficulty in interpreting the results---pale in comparison to these benefits.

%%% Local Variables:
%%% mode: latex
%%% TeX-master: "doe"
%%% End:



\section{The Capability Ratio and Its Discontents}

Thanks in part to the popularity of formal models of choice under uncertainty, many unobserved quantities like those discussed above take the form of probabilities.
Our application---expectations about war outcomes as parameterized by some probability $p \in [0,1]$---is no different.
We want to create a proxy for the chance that Country~A would prevail in a dispute against Country~B, given their observable characteristics, $x_A$ and $x_B$.
Since a measure of a probability must lie within the unit interval, a natural way to proceed is to propose an indexing function $g$, where $g(x) \geq 0$, and then take the ratio of indices,
\begin{equation}
  \label{eq:ratio}
  f(x_A, x_B)
  =
  \frac{g(x_A)}{g(x_A) + g(x_B)}.
\end{equation}
The quality of such a measure depends on both the selected characteristics and on the appropriateness of the indexing function~$g$.
This latter responsibility plays a large role in the development of good measures and is our primary area of focus.

Though simple, this enhanced ratio-based approach is remarkably powerful and finds use in a diverse array of applications.
A classic success comes from the study of baseball outcomes, where the Pythagorean prediction \citep{james1983,miller2007} of a team's winning percentage is defined as
\begin{align*}
  f\left(\text{Runs Scored}, \text{Runs Allowed} ; \alpha\right) &= \frac{\text{Runs Scored}^\alpha}{\text{Runs Scored}^\alpha + \text{Runs Allowed}^\alpha},
\end{align*}
where $\alpha \geq 0$ adjusts $x$'s shape.
Here the quest for the best-fitting $g$ amounts to estimating $\alpha$; \citet{james1983} originally proposed $\alpha = 2$ \emph{ad hoc}, and later analysts found that $\alpha = 1.83$ fit the data best.
Though the analyses that produced this estimate suffer from the overfitting problems discussed above, the Pythagorean predictor still performs quite well when imposed upon out of sample data.

% TODO: Some of this notation collides with what we use for multiple imputation in the appendix.  Do we care?

When proxying for expected dispute outcomes, empirical conflict scholars normally use transformations of data on states' material capabilities.
We now relate the typical transformation to our discussion of ratio-based measures above.
We begin by introducing some helpful notation: call the set of states $\mathcal{I} = \left\{1, \ldots, I\right\}$; the set of variables $\mathcal{J} = \left\{1,\ldots,J\right\}$; and the set of years $\mathcal{T} = \left\{1,\ldots, T\right\}$.
Denote state $i$'s value for variable $j$ in time $t$ as $M_{ijt}$.
The set of all data is $M$, and all data in year $t$ is $M_t$.
Define state $i$'s share of variable $j$ in year $t$ as
\begin{align*}
  S_{ijt}(M_{t}) &= \frac{M_{ijt}}{\sum_{\mathcal{I}} M_{ijt}}.
\end{align*}

We now introduce the \emph{CINC function}.\footnote{Here CINC stands for ``Composite Index of National Capability'' as given in the Correlates of War National Material Capabilities data \citep{singer1972}.}
State $i$'s CINC score in year $t$ is its average share across all variables:
\begin{align*}
  \CINC_i(M_t) &= \frac{\sum_{\mathcal{J}} S_{ijt}}{\left|\mathcal{J}\right|}.
\end{align*}
State $i$'s CINC score therefore falls in $[0,1]$.
The CINC score is the ``most commonly used measure'' of power in empirical conflict studies \citep[212]{kadera2004}.

Following the discussion in the previous section, the most intuitive CINC-based proxy for $p$ is a na\"{\i}ve \emph{capability ratio}:
\begin{align*}
  f_{\text{CR}}(M_t) &= \frac{\CINC_A(M_t)}{\CINC_A(M_t) + \CINC_B(M_t)}.
\end{align*}
Our approach, then, makes explicit the fact that $\CINC$ is simply a candidate $g$ function imposed upon annual material capability data $M_t$.

Problems emerge immediately.
Two of these pertain to the CINC function itself.
First, as has been documented, the CINC function is sensitive to changes in state membership over time \citep{organski1980,gleditsch1999,kadera2004}.
Second, the CINC function's equal weighting of all indicators is entirely \emph{ad hoc}.
For example, the CINC function assigns the same importance to military spending as it does to personal energy consumption.
Whether this is an appropriate assignment is an empirical question that goes unanswered.
Even on the tenuous assumption that $\CINC$ is a good data reduction technique on $M_t$, it is not clear whether it serves as a good $g$ function.
Prior to entering $f_{\text{CR}}$, should the CINC scores be exponentiated given some parameter on returns to scale, or perhaps instead logged?
Finally, even given that $\CINC$ is a useful index, we do not know whether a ratio-based approach is appropriate at all.
In other words, taking the capability ratio at its word requires making a host of assumptions that may not hold well enough to make it useful in applications.\footnote{
  It is worth noting that some applications follow \citet{bremer1992} in using the explicit CINC ratio:
  \begin{align*}
    f_{\text{Bremer}}(M_t) &= \frac{\max\left\{\CINC_A(M_t),\CINC_B(M_t)\right\}}{\min\left\{\CINC_A(M_t),\CINC_B(M_t)\right\}}.
  \end{align*}
  Bremer's approach does not fall in $[0,1]$, though it could be transformed through a logit or probit CDF.
  However, it is a monotonic transformation of the aforementioned capability ratio, and it suffers from similar problems anyway.
}

Yet it is widely used. Capability ratios (or similar manipulations of CINC scores) feature prominently in many recent empirical studies in international relations.
As we might expect given the importance of the bargaining model, many of these \citep[e.g.][]{gartzke2007,salehyan2008} use capability ratios in regressions predicting the onset of a militarized interstate dispute.
Still others focus on particular features of a militarized interstate dispute, such as the nature of its termination \citep{beardsley2008} or whether its combatants complied with laws of war \citep{morrow2007}.
Still other studies focus on other phenomena not directly related to disputes, such as the onset of sanctions \citep{whang2013}, issue agreements \citep{mitchell2007}, or nuclear assistance provisions \citep{kroenig2009}.
A more exhaustive survey of the use of capability ratios is beyond the scope of this paper, but suffice it to say that it is the go-to measure of relative power in international relations.

One might politely defend the capability ratio by noting that it asks the $\CINC$ function to perform a job it was not designed for.
We would agree.
It is worth noting, however, that early proponents of the $\CINC$ function \citep[24]{singer1972} sought to understand how ``uncertainty links[s] up with capability patterns on the one hand and with war or peace on the other.''
Writing over two decades before the classic introduction to the bargaining model of war \citep{fearon1995}, these authors lacked the abstract target---$p$---that we currently have, but their enterprise was largely similar.
Though their focus on systemic, rather than dyadic, patterns reflects the dominant flavor of realism at the time, they still wanted to know how preponderance of power related to the decision to fight.
So, while it remains true that the capability ratio is not meant to directly relate to $p$ in a bargaining model, it \emph{was} meant to tell us something about how uncertainty relates to war.

Remedying these problems by fitting a model---perhaps estimating an exponent on the CINC scores in $f_{\text{CR}}$, or weights for the CINC indicators---would be a laudable first step, but it would still suffer from functional form dependencies.
The careful scholar might sidestep these by fitting an ensemble of models with different functional forms imposed, but so long as the ensemble is fit to the entire capability data set, it will suffer from overfitting and lose predictive power.
In the next section, we outline our method for estimating $p$ that suffers neither from overfitting nor from pathologies in the $\CINC$-utilizing, ratio-based approach.

%%% Local Variables:
%%% mode: latex
%%% TeX-master: "doe"
%%% End:



\section{Building a Better Proxy for Expected Dispute Outcomes}
\label{sec:methods}

Our goal now is to squeeze as much predictive power as we can from data on states' material capabilities.
When prediction is the goal, ``black box'' algorithmic techniques usually outpace standard regression models \citep{Breiman:2001fd}.
So, to build our new measure, we augment traditional approaches with methods from machine learning.

\subsection{Data}

To evaluate the predictive performance of the capability ratio and then to build an alternative measure, we use data on the outcomes of international disputes.
We combine the National Material Capabilities data \citep{singer1972} with information on the outcomes and participants of Militarized International Disputes between 1816 and 2007 \citep{Palmer:2015hp}.
Our data consist of $N = 1{,}740$ disputes, each between an ``initiator,'' or Country~A, and a ``target,'' or Country~B.\footnote{
  See the Appendix for the data construction and coding specifics.
}
Every dispute outcome is either A~Wins, B~Wins, or Stalemate, which we denote by $Y_i \in \{A, B, \emptyset\}$, respectively.
Most disputes end in a stalemate, and victory by the initiator is more than twice as likely as victory by the target, as shown in Table~\ref{tab:mid}.

\begin{table}[htp]
  \centering
  \input{tab-mid}
  \caption{
    Distribution of the three dispute outcomes.
  }
  \label{tab:mid}
\end{table}

We model dispute outcomes as a function of the participants' military capabilities.
Our data source, the National Material Capabilities dataset, records annual observations of six characteristics of a country's military capability: military expenditures, military personnel, iron and steel production, primary energy consumption, total population, and urban population.\footnote{
  There are missing observations in the National Material Capabilities data.
  Consequently, about 17~percent of the disputes we observe contain at least one missing cell.
  We use multiple imputation to deal with missingness \citep{honaker_what_2010}; see the Appendix for details.
}
We also calculate each country's share of the global total of each component, giving us 12 variables per dispute participant.
The matrix of predictors has 26 columns: the 24 individual capability characteristics of the initiator and target, the standard capability ratio, and the year the dispute began.
The values of these predictors for the $i$'th dispute are collected in the vector $X_i$.

\subsection{A Metric for Predictive Power}

We face two challenges in evaluating a model's predictive power.
The first is to define a metric for predictive power---one that is appropriate to the task at hand and reasonably interpretable.
The second is to measure each model's ability to predict \emph{out of sample}.
Our main purpose, which is to measure the chances of victory for each side in a hypothetical interstate dispute, is inherently an out-of-sample prediction task.
We do not want a model that overfits the sample data at the expense of its predictive power when brought to new data.

As fortune plays a role in every military engagement, it would be impossible to perfectly predict the outcome of every dispute.
We therefore want a measure of predictive power that respects the probabilistic nature of militarized disputes.
Classification metrics like the accuracy statistic, also known as the percentage correctly predicted, do not fit the bill.
Instead, we employ the log loss, which is the negative of the average log-likelihood, as our metric for predictive power \citep[221]{Hastie:2009wpa}.
Let a \emph{model} be a function $\hat{f}$ that maps from the dispute-level predictors $X_i$ into the probability of each potential dispute outcome, $\hat{f}(X_i) = (\hat{f}_A(X_i), \hat{f}_B(X_i), \hat{f}_{\emptyset}(X_i))$.
The ``hat'' on $\hat{f}$ is there to emphasize that the form of the function has been learned from the data, whether by estimating regression coefficients or by a more flexible predictive algorithm.
The log loss of the model~$\hat{f}$ on the data~$(X, Y)$ is\footnote{
  To avoid numerical problems, very low probabilities are trimmed at $\epsilon = 10^{-14}$.
}
\begin{equation}
  \label{eq:log-loss}
  \ell(\hat{f}, X, Y)
  =
  - \frac{1}{N} \sum_{i = 1}^{N} \sum_{t \in \{A, B, \emptyset\}}
  \mathbf{1} \{Y_i = t\} \log \hat{f}_t(X_i).
\end{equation}
Smaller values of the log loss represent better predictive power, with the lower bound of~$0$ indicating perfect prediction.

We care mainly about the generalization error of our models---the expected quality of their predictions for new data that was not used to fit the models.
Our small sample size of $N = 1{,}740$ makes this tricky.
If we had a surplus of observations, we could use some suitably large number to fit our models and hold out the remainder to assess the models' predictive power.
But with as little data as we have, splitting the sample is ill-advised: we cannot hold out enough observations to estimate the generalization error precisely without harming the precision of the model itself.
So, to measure out-of-sample predictive power without losing data, we turn to $K$-fold cross-validation \citep[241--249]{Hastie:2009wpa}.
We randomly assign each dispute observation to a ``fold'' $k \in \{1, \ldots, K\}$, where we follow standard practice by setting $K = 10$.\footnote{Standard practice here stands on firm ground; \citet{molinaro2005} find that 10-fold cross-validation performs quite similarly to leave-one-out cross validation (the ``ideal case'' for cross-validation) without having to take on massive computational costs.  10-fold cross-validation also performs better than the .632+ bootstrap, split-sample techniques, and Monte Carlo cross-validation, particularly in smaller samples like ours.}
For each~$k$, we split the data into a ``test'' sample containing fold~$k$ and a ``training'' sample containing the remainder of the data.
We fit a model only on the training sample and then calculate its predicted probabilities for the data in the test sample.\footnote{
  \label{fn:nested-cv}
  When dealing with models with tuning parameters that are themselves selected by cross-validation, we choose tuning parameters separately within each of the $K$ iterations via another cross-validation loop.
  This nested cross-validation is necessary to keep our estimates of generalization error from being too optimistic \citep{Varma:2006ch}.
}
After repeating this $K$ times, we have an out-of-sample prediction for each observation in our data---one that was calculated from a model that did not see the observation in question.
We compare these predicted probabilities to the observed outcomes to estimate our models' generalization error.
Formally, the cross-validation loss of the model~$\hat{f}$ is the average out-of-fold log loss,
\begin{equation}
  \label{eq:cv-loss}
  \CVL(\hat{f})
  =
  \frac{1}{K} \sum_{k=1}^K \ell \left(
    \hat{f}^{(-k)}, X^{(k)}, Y^{(k)}
  \right),
\end{equation}
where $(X^{(k)}, Y^{(k)})$ is the data in the $k$'th fold and $\hat{f}^{(-k)}$ is the model~$\hat{f}$ fit to the data excluding the $k$'th fold.

Because it is measured on the log-likelihood scale, the log loss metric is hard to interpret on its own.
To ease the interpretation, we compare models' log loss to that of a null model, whose predicted probabilities always equal the sample proportions of each outcome.
The proportional reduction in cross-validation loss of the model~$\hat{f}$ is
\begin{equation}
  \label{eq:prl}
  \PRL(\hat{f})
  =
  \frac{
    \CVL(\hat{f}_{\text{null}}) - \CVL(\hat{f})
  }{
    \CVL(\hat{f}_{\text{null}})
  }.
\end{equation}
The theoretical maximum, for a model that predicts perfectly, is~$1$.
If a model predicts even worse than the null model---meaning, in essence, it is worse than random guessing---its proportional reduction in loss is negative.

\subsection{Modeling Dispute Outcomes}

Our task now is twofold: to assess the predictive power of the capability ratio and, should we find it lacking (as we do), to build a better alternative.

We model dispute outcomes as a function of the capability ratio via ordered logistic regression \citep{McKelvey:2010gv}.
To reduce skewness, we take the natural logarithm of the capability ratio.
The parameter estimates from the capability ratio model on the full sample appear in Table~\ref{tab:capratio}.
Although these results do not speak directly to the capability ratio's out-of-sample performance, they foreshadow why its predictive power is so limited.
The coefficient on the capability ratio is statistically significant but small relative to the cutpoints, indicating a substantively weak relationship.
Dividing the cutpoints by the coefficient, we see that we would need a logged capability ratio below $-13$ or above $+7$ to predict any outcome other than a stalemate.
These bounds lie well outside the observed range of capability ratios in the dispute data, which are bounded below by $-9.1$ (Palau--Philippines 2000) and above by $-0.0004$ (Germany--Panama 1940).
In other words, the capability ratio always predicts a stalemate within the sample.
This does not bode well for its out-of-sample performance.

\begin{table}[tp]
  \centering
  \input{tab-capratio}
  \caption{
    Results of an ordered logistic regression of dispute outcomes on the capability ratio using the training data.
    Because there are no missing values in the CINC scores, these estimates are identical across imputed datasets.
  }
  \label{tab:capratio}
\end{table}

We want a better model than what the capability ratio gives us, but we do not have a strong \emph{a priori} sense of what the data-generating process---the true relationship between material capabilities and dispute outcomes---looks like.
So we use tools from machine learning that are designed to predict well without imposing much structure on the data.
Ideally, we would select the predictive model that is best for our data, but there are too many algorithms to try them all.
To narrow it down, we defer to the machine learning experts on which algorithms are best.
We draw our set of candidate models from the top-ten list by \citet{Wu:2007ev} and from the best performers in the tests by \citet{FernandezDelgado:2014ul}.
After excluding those unsuited to our data,\footnote{
  Four of the algorithms named in \citet{Wu:2007ev}---$k$-means, Apriori, expectation maximization, and PageRank---are not suited for the prediction task at hand.
  We also excluded AdaBoost due to long computation time and naive Bayes due to poor performance in initial tests.
}
we end up with six predictive algorithms to try: C5.0, support vector machines, $k$-nearest neighbors, classification and regression trees, random forests, and ensembles of neural nets.\footnote{
  See the Appendix for full details of each method.
}
Each algorithm is widely used for prediction and can predict dispute outcome probabilities as a complex, potentially nonlinear function of the material capability components.
As a compromise between these flexible ``black box'' models and the rigid capability ratio model, we also test ordered logistic regression models on the capability components.

In the spirit of flexibility, we try each model with different sets of predictors from the capability data.
We examine four sets of variables: the raw capability components and the annual component shares, each with and without the year the dispute began.
All of our models allow for interactive relationships, so including the year of
the dispute lets the effect of each capability component vary over time.
With two sides per dispute and six capability variables per side, each model has 12 or 13 variables, depending on whether the year is included.
All told, we have 30 candidate models: four sets of variables for each of our seven algorithms, plus the capability ratio model and a null model used as a baseline.

We use cross-validation to estimate how well each of our candidate models predicts out of sample.
The final problem, once we have the cross-validation results, is to choose a model---the one we will use to construct an alternative to the capability ratio as a measure of expected dispute outcomes.
It is tempting to simply pick the model with the lowest cross-validation loss.
We can do even better at prediction, however, by taking a weighted average of all the models.
We use the super learner algorithm \citep{vanderLaan:bz} to select the optimal model weights via cross-validation.
Given a set of $M$ candidate models $\hat{f}_1, \ldots, \hat{f}_M$, we select weights $\hat{w}_1, \ldots, \hat{w}_M$ to solve the constrained optimization problem
\begin{equation}
  \label{eq:super-learner}
  \begin{aligned}
    \min_{w_1, \ldots, w_M}
    &\quad
    \CVL \left(
      \sum_{m=1}^M w_m \hat{f}_m
    \right)
    \\
    \mbox{s.t.}
    &\quad
    w_1, \ldots, w_m \geq 0,
    \\
    &\quad
    w_1 + \ldots + w_m = 1,
  \end{aligned}
\end{equation}
Our final model is the super learner, $\hat{f} = \sum_m \hat{w}_m \hat{f}_m$.
Each individual model is a special case of the super learner, with full weight $\hat{w}_m = 1$ placed on a single $\hat{f}_m$.
Hence, by the cross-validation criterion, we should prefer the super learner over any individual model.\footnote{
  \label{fn:sl-bias}
  As usual when selecting tuning parameters via cross-validation, the value of equation~\eqref{eq:super-learner} is not an unbiased estimate of the generalization error of the super learner.
  Nested cross-validation is computationally infeasible for the super learner, so we calculate the bias correction recommended by \citet{Tibshirani:2009tz} to estimate its generalization error.
}  That said, the super learner does provide the capability ratio with an opportunity to defend itself; should it earn a high weight, then our costly enterprise may not be worth the effort.

To summarize, we fit and cross-validate $M = 30$ candidate models, then combine them into a super learner that we will use to construct a better proxy for expected dispute outcomes.
The biggest downside of our approach is that the results are not easily interpretable.
Because the super learner entails averaging a large set of models---some of which, like random forests, are themselves difficult to interpret---it gives us no simple summary of how each predictor affects dispute outcomes.
This is not a problem, given our aims.
Certainly, we would not recommend the super learner as a means of testing hypotheses about the determinants of dispute outcomes.
However, our goal is not to test a hypothesis---it is to construct the best proxy possible for how a dispute between two countries is likely to end.
In this context, it is worth sacrificing interpretability for the sake of predictive power.

\subsection{Results}

We now turn to the cross-validation results, which are summarized along with the super learner weights in Table~\ref{tab:ensemble}.
As the in-sample analysis hinted, the capability ratio is indeed a poor predictor of dispute outcomes.
Its proportional reduction in loss is~0.01, which means its predicted probabilities are just 1~percent more accurate than the null model.
This number is not encouraging, but what matters even more is whether we can do better.
A glance at Table~\ref{tab:ensemble} confirms that we can: all but one of our 27 alternative models has greater predictive power than the capability ratio, many of them considerably better.
With these results in hand, we feel comfortable dismissing the capability ratio as a suboptimal proxy for expected dispute outcomes.

\begin{table}[tp]
  \centering
  \input{tab-ensemble}
  \caption{
    Summary of cross-validation results and super learner weights.
    All quantities represent the average across imputed datasets.
  }
  \label{tab:ensemble}
\end{table}

We can glean a few basic intuitions about the relationship between capabilities and dispute outcomes from the cross-validation results.
First, the relationship is too complex to capture in a linear model.
The best of our ordered logits has a 10~percent proportional reduction in loss, about half that of the best nonlinear model (the neural net ensemble on capability components with year included).
So we cannot fix the problems with the capability ratio just by disaggregating it into its component parts.
Second, the capability--outcome relationship changes over time.
Accounting for the year of the dispute improves our models' predictive power: in 13 out of 14 cases, the model that includes time predicts better than its time-less counterpart.\footnote{
  The difference in log loss is statistically significant (paired $t = -3.25$, $p = 0.006$).
}
Unlike the capability ratio, flexible predictive algorithms can capture this variation.
Finally, the models trained on the raw capability components tend to outperform those trained on the annual shares of components.\footnote{
  The difference in log loss is statistically significant (paired $t = -3.14$, $p = 0.008$).
}
This runs contrary to way the CINC score is constructed---it is an average of capability shares---suggesting another reason why the capability ratio is such a poor predictor.

As we expected, the super learner ensemble performs better than any of the candidate models from which it is constructed.
The ensemble's proportional reduction in loss is about 23~percent, or four percentage points better than the best candidate model.
Even after we apply a bias correction (see footnotes~\ref{fn:nested-cv} and~\ref{fn:sl-bias}), the super learner's predictive power is still the best among our models.
Looking at the weights, what stands out is how few models are substantial components of the super learner: just five models have a weight of at least 5~percent.
More generally, while models with lower generalization error tend to receive more weight, the relationship is by no means one-to-one.
We see this because the ensemble prefers not only predictive power, but also diversity.
Different classes of models have different blind spots; the more diverse the ensemble is, the more these blind spots are minimized.
A model that looks bad on its own might still merit non-negligible weight in the optimal ensemble if it captures a slice of the data missed by the models that are best on their own.

\begin{figure}[tp]
  \centering
  \input{fig-oof-pred}
  \vspace{-2em}
  \caption{
    Ternary plots of out-of-fold predicted probabilities according to the capability ratio model and the super learner.
    Each predicted probability is calculated by fitting the model to 9/10 of the data, not including the observation in question---an approach that simulates true out-of-sample prediction.
  }
  \label{fig:oof-pred}
\end{figure}

The super learner predicts dispute outcomes much better than the capability ratio does.
As we have just shown, the capability ratio only improves by~1~percent on a null model, whereas the super learner gives a~20~percent improvement.
For another illustration of the difference in predictive power, see the plots of out-of-fold predicted probabilities---the ones we use in cross-validation---in Figure~\ref{fig:oof-pred}.
Under the capability ratio model, all but a handful of disputes are predicted to have an 80--90~percent chance of ending in stalemate.
Seeing how narrow the capability ratio's predictive range is, it is little surprise that it barely does better than a null model at prediction.
Conversely, the super learner makes much better use of the material capability data.
Its predictive range is greater, which in turn allows it to achieve a stronger---though hardly perfect---relationship between predicted and observed outcomes.

What should we make of the fact that, even after all this predictive effort, about 80~percent of the variation in dispute outcomes remains unexplained?
One problem is that our sample size is too small to fit flexible models without a loss of precision.
For the sake of precise predictions---though not that of humankind!---it would be better to have far more dispute observations to use for model building.
The bigger issue, however, is that material capabilities do not tell the whole story.
No matter how many modeling tricks we pull out, we cannot change the fact that militarized dispute outcomes depend on much more than raw capabilities.
Some of these other predictors, like the distance between the disputants, are easily measured; others, like the cleverness of their leaders, are not.
In principle, it is possible to incorporate additional variables (at least those that can be quantified) into the super learner to attain even better outcome predictions.
That goes beyond the scope of this paper, which only aims to show how we can best use the data underlying the capability ratio.

%%% Local Variables:
%%% mode: latex
%%% TeX-master: "doe"
%%% End:



\section{Measurement Validity and the Empirical Study of Conflict}


\section{Conclusion}


\clearpage
\appendix
\section{Appendix}

\subsection{National Material Capabilities Data}

Our predictors are taken from the National Material Capabilities (v4.0) dataset from the Correlates of War project \citep{singer1972}.\footnote{
  Downloaded from \url{http://correlatesofwar.org/data-sets/national-material-capabilities/nmc-v4-data/at_download/file}.
}
The dataset contains observations on six variables for 14,199 country-years from 1816 to 2007.
For details on the variables and their measurement, see the NMC Codebook.\footnote{
  Available at \url{http://correlatesofwar.org/data-sets/national-material-capabilities/nmc-codebook/at_download/file}.
}
Table~\ref{tab:summary} lists the proportions of zeroes and missing values among each variable.

\begin{table}[htp]
  \centering
  \input{tab-summary}
  \caption{
    Proportions of zeroes and missing values in each National Military Capability component variable.
  }
  \label{tab:summary}
\end{table}

All six variables are strongly right-skewed.
Since five of the six variables are sometimes zero-valued (though all are non-negative), a logarithmic transformation is not appropriate.
Instead, to correct for skewness, we apply an inverse hyperbolic sine transformation \citep{Burbidge:1988gu} to each component:
\begin{equation}
  \label{eq:asinh}
  h(x, \theta)
  =
  \sinh^{-1} (\theta x)
  =
  \log \left(
    \theta x + \sqrt{(\theta x)^2 + 1}
  \right).
\end{equation}
We set the scale~$\theta$ separately for each component variable with the aim of making the transformed variable approximately normally distributed.
For each variable, we choose the value of $\theta \in \{2^d\}_{d=-10}^{10}$ that minimizes the Kolmogorov--Smirnov test statistic \citep{MasseyJr:2012jo} against a normal distribution with the same mean and variance.
Table~\ref{tab:summary} gives the scale selected for each component.
We use the transformed components in both the multiple imputation (see below) and the super learner training.

\subsection{Militarized Interstate Dispute Data}

Our sample and outcome variable are taken from the Militarized Interstate Disputes (v4.1) dataset from the Correlates of War project \citep{Palmer:2015hp}.\footnote{
  Downloaded from \url{http://correlatesofwar.org/data-sets/MIDs/mid-level/at_download/file}.
}
The dataset records the participants and outcomes of interstate disputes from 1816 to 2010.
To avoid the problem of aggregating capabilities across multiple states, we exclude disputes with more than one state on either side.
We drop disputes that end in an outcome other than one side winning, one side yielding, or a stalemate;\footnote{
  For details on other kinds of outcomes, see the MID Codebook.
}
we then collapse ``A Wins'' and ``B Yields'' into a single coding, and similarly for ``B Wins'' and ``A Yields.''
Finally, since the capabilities data only run through 2007, we exclude disputes that end after 2007.
In the end, we have $N = 1{,}740$ cases.

For each dispute in our dataset, we code the participating countries' capabilities using the values in the year the dispute began.
About 17~percent of disputes have at least one missing capability component for at least one participant.

\subsection{Multiple Imputation}

As noted above, all of the National Material Capabilities variables contain some missing values.
Following standard practice, we multiply impute the missing observations.
We perform the imputations via the \texttt{Amelia} software package \citep{pkg-Amelia}.

Rather than just impute the missing values in the final dataset of disputes, we impute the entire National Material Capabilities dataset.
This allows us to fully exploit the dataset's time-series cross-sectional structure in the imputation process \citep{honaker_what_2010}.
We include in the imputation model a cubic polynomial for time, interacted with country dummy variables.
As this results in an explosion in the number of parameters in the imputation model, we then impose a ridge prior equal to 0.1 percent of the observations in the dataset (see Section~4.7.1 of the \texttt{Amelia} package vignette).
We enforce the constraint that every imputed value be non-negative.
Finally, we impose an observation-level prior with mean zero and variance equal to that of the observed values of the corresponding component variable for every missing cell that meets the following criteria:
\begin{itemize}
  \item There are no non-zero observed values in the time series preceding the cell
  \item The first observed value that comes after the cell is zero
\end{itemize}
So, for example, if a country's urban population is zero from 1816 to 1840, missing from 1841 to 1849, and zero in 1850, we would impose this form of prior on the 1841--1849 values.
Diagnostic time series plots of observed versus imputed values within each data series, generated by the \texttt{tscsPlot()} function in \texttt{Amelia}, will be made available in the project's Dataverse.

The presence of missing data also complicates the calculations of country-by-country proportions of the total amount of each component by year.
One option is to recompute the annual totals in each imputed dataset, so that the resulting data will be logically consistent---in particular, all proportions will sum to one.
The drawback of this approach is that virtually every observation of the proportions will differ across the imputed datasets, even for countries with no missing data, since the annual totals will differ across imputations.
An alternative approach is to compute the annual totals using only the observed values.
The advantage is that non-missing observations will not vary across imputed datasets; the downside is that the proportions within each imputation will generally sum to more than one.
For our purposes in this paper, we think it is preferable to reduce variation across imputations, even at the expense of some internal consistency in the imputed datasets, so we take the latter approach: annual totals are the sums of only the observed values.

We impute $I = 10$ datasets of national capabilities according to the procedure laid out above, and we merge each with the training subset of our dispute data to yield $I$ training data imputations.
We run the super learner separately on each imputation, and our final model is an (unweighted) average of the $I$ super learners.

After training is complete, we run into missing data problems once again when calculating DOE scores.
To calculate predicted probabilities for dyads with missing values, we calculate a \emph{new} set of $I = 10$ imputations of the capabilities data and take an (unweighted) average of our model's predictions across the imputations.

\subsection{Super Learner Candidate Models}

We use the R statistical environment \citep{pkg-R} for all data analysis.
We fit, cross-validate, and calculate predictions from each candidate model through the \texttt{caret} package \citep{pkg-caret}.
We then construct the super learner by solving~\eqref{eq:super-learner} via base R's \texttt{constrOptim()} function for optimization with linear constraints.
Further details about each candidate model are summarized below.

\begin{itemize}
  \item Ordered Logit \citep{McKelvey:2010gv}
  \begin{description}
    \item[Package] \texttt{MASS} \citep{pkg-MASS}
    \item[Tuning Parameters] None
    \item[Notes] In the ``Year'' models, the year of the dispute is included directly and interacted with each capability variable
  \end{description}

  \item C5.0 \citep{Quinlan:2015uc}
  \begin{description}
    \item[Package] \texttt{C50} \citep{pkg-c50}
    \item[Tuning Parameters] ~
    \begin{itemize}
      \item Number of boosting iterations (\texttt{trials}): selected via cross-validation from $\{1, 10, 20, 30, 40, 50\}$
      \item Whether to decompose the tree into a rule-based classifier (\texttt{model}): selected via cross-validation
      \item Whether to perform feature selection (\texttt{winnow}): selected via cross-validation
    \end{itemize}
  \end{description}

  \item Support Vector Machine \citep{Cortes:1995ie}
  \begin{description}
    \item[Package] \texttt{kernlab} \citep{pkg-kernlab}
    \item[Tuning Parameters] ~
    \begin{itemize}
      \item Kernel width (\texttt{sigma}): selected via cross-validation from $\{0.2, 0.4, 0.6, 0.8, 1\}$
      \item Constraint violation cost (\texttt{C}): selected via cross-validation from $\{\frac{1}{4}, \frac{1}{2}, 1, 2, 4\}$
    \end{itemize}
    \item[Notes] Radial basis kernel
  \end{description}

  \item $k$-Nearest Neighbors \citep{Cover:1967jq}
  \begin{description}
    \item[Package] \texttt{caret} \citep{pkg-caret}
    \item[Tuning Parameters] ~
    \begin{itemize}
      \item Number of nearest neighbors to average (\texttt{k}): selected via cross-validation from $\{25, 50, \ldots, 250\}$
    \end{itemize}
    \item[Notes] All predictors centered and scaled to have zero mean and unit variance
  \end{description}

  \item CART \citep{Breiman:1984tu}
  \begin{description}
    \item[Package] \texttt{rpart} \citep{pkg-rpart}
    \item[Tuning Parameters] ~
    \begin{itemize}
      \item Maximum tree depth (\texttt{maxdepth}): selected via cross-validation from $\{2, 3, \ldots, 9, 10\}$ (only up to 9 for models without year included)
    \end{itemize}
  \end{description}

  \item Random Forest \citep{Breiman:2001fb}
  \begin{description}
    \item[Package] \texttt{randomForest} \citep{pkg-randomForest}
    \item[Tuning Parameters] ~
    \begin{itemize}
      \item Number of predictors randomly sampled at each split (\texttt{mtry}): selected via cross-validation from $\{2, 4, \ldots, 12\}$
    \end{itemize}
    \item[Notes] 1,000 trees per fit
  \end{description}

  \item Averaged Neural Nets \citep{Ripley:1996vd}
  \begin{description}
    \item[Package] \texttt{nnet} \citep{pkg-MASS}, \texttt{caret} \citep{pkg-caret}
    \item[Tuning Parameters] ~
    \begin{itemize}
      \item Number of hidden layer units (\texttt{size}): selected via cross-validation from $\{1, 3, 5, 7, 9\}$
      \item Weight decay parameter (\texttt{decay}): selected via cross-validation from $\{10^0, 10^{-1}, 10^{-2}, 10^{-3}, 10^{-4}\}$
    \end{itemize}
    \item[Notes] Creates an ensemble of 10 neural nets, each initialized with different random number seeds
  \end{description}
\end{itemize}

\subsection{Replications}

The following list contains basic information about each model in the replication study.
We carry out logistic and probit regressions via \texttt{glm()} in base R \citep{pkg-R}, multinomial logit via \texttt{multinom()} in the \texttt{nnet} package \citep{pkg-MASS}, ordered probit via \texttt{polr()} in the \texttt{MASS} package \citep{pkg-MASS}, and heteroskedastic probit via \texttt{hetglm()} in the \texttt{glmx} package \citep{pkg-glmx}.

\input{list-replications}

%%% Local Variables:
%%% mode: latex
%%% TeX-master: "doe"
%%% End:


\newpage
\bibliographystyle{apsr}
\bibliography{doebib}


\end{document}
